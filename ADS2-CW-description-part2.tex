\documentclass[11pt]{article}
\usepackage[colorlinks = true]{hyperref}
\usepackage{geometry}                % See geometry.pdf to learn the layout options. There are lots.
\geometry{letterpaper}                   % ... or a4paper or a5paper or ... 
%\geometry{landscape}                % Activate for for rotated page geometry
%\usepackage[parfill]{parskip}    % Activate to begin paragraphs with an empty line rather than an indent
\usepackage{graphicx}
\usepackage{amssymb}
\usepackage{epstopdf}
\DeclareGraphicsRule{.tif}{png}{.png}{`convert #1 `dirname #1`/`basename #1 .tif`.png}

\title{ADS2 Assessed Exercise: Modeling a Cache -- Part II}
\author{Wim Vanderbauwhede}
\date{\vspace{-5ex}}                                           % Activate to display a given date or no date

\begin{document}
\maketitle

\section{Aim}\label{aim}

The aim of this coursework is to create a Java model for a simple cache as used in modern computer systems.

\section{What to submit}\label{what-to-submit}

For the second part of the assignment, you have to write the code for the \texttt{FullyAssocLiFoCache} class based on the provided skeleton and the required Java datastructures.
\begin{itemize}
\item You \emph{must} start from the provided code and you should \emph{only add your own code} in \texttt{FullyAssocLiFoCache.java}, you \emph{must} not modify or removed any part of the provided code. 
\item You \emph{must} use the \texttt{Status} class for status reporting as it will be used by the testbench to verify the correctness of your code.
\item If you use \texttt{println } statements in your code, they \emph{must} be guarded by an \texttt{if(VERBOSE)} guard, for example:
\begin{verbatim}
   if (VERBOSE) System.out.println("Flushing cache");
\end{verbatim}

\item You \emph{must} submit this code in a \emph{gzipped tar archive} (other formats will \emph{not} be accepted) through
the Moodle submission system, and the filename \emph{must} be
\emph{\textless{}your matric + 1st char of your name in lowercase\textgreater{}.tgz}, so for example if your matric number is \emph{1107023m}
then your archive \emph{must} be named \emph{1107023m.tgz}. This archive
\emph{must} contain a single folder named \emph{\textless{}your matric + 1st char of your name in lowercase\textgreater{}}. This folder
\emph{must} contain a subfolder \texttt{src/ads2/cw1/} with following files:
\begin{itemize}
\item \texttt{Cache.java} as provided
\item \texttt{FullyAssocLiFoCache.java}  in which you must implement the functionality for the cache model
\item \texttt{Main.java} as provided
\item \texttt{Memory.java} as provided
\item \texttt{Status.java} as provided
\item \texttt{TestBench.java} as provided
\end{itemize}
\end{itemize}

\section{What is provided}\label{what-is-provided}

You can get the coursework source files and the coursework description at


\section{How to test your code}\label{how-to-test-your-code}

There are several ways to test your code:

\section{Marking}\label{marking}



Your code will be marked using a test script. This script will test if your code builds correctly and runs correctly for a number of use cases. I will not modify the testbench code but may modify the memory and cache parameters. I will also run a few additional tests.
The script will also perform source code analysis to see if you have used the correct API calls, control structures etc. 

\end{document}  